Application pour les employés de la maison des ligues permettant la gestion et l\textquotesingle{}affichage de l\textquotesingle{}espace de formation. Cette application a pour titre de faciliter l\textquotesingle{}affichage et le listage des stages de formations sur une période de temps (planning).

\subsection*{Interface de connexion}

Interface de connexion simple permettant la saisie de l\textquotesingle{}identifiant (e-\/amil) et mot de passe de l\textquotesingle{}utilisateur. Le mot de passe utilise le protocole de cryptage {\bfseries S\+H\+A1}.



\subsection*{Menu}

Cette application dispose d\textquotesingle{}un menu après connexion, permettant à l\textquotesingle{}utilisateur de choisir l\textquotesingle{}action attendue. L\textquotesingle{}utilisateur a le choix entre consulter les {\itshape stages de formations à venir} ou bien {\itshape consulter toutes les formations} disponibles.

\subsection*{Formations à venir}

L\textquotesingle{}utilisateur connecté peut consulter les stages de formations à venir par défaut d\textquotesingle{}une date allant du jour J au jour J+7.



Afin de permettre la recherche de stages de formations par date, l\textquotesingle{}utilisateur dispose d\textquotesingle{}une fonction de {\bfseries recherche avancée}. Celle-\/ci fonctionne au travers de différent élément clé tel qu\textquotesingle{}une fourchette de date ou autre, recupéré afin de composer une requete S\+QL permettant son affichage filtré à l\textquotesingle{}aide de la fonction suivante \+:


\begin{DoxyCode}
1 dbConnect.listViewStagesFormations(query, listViewStagesFormations);
\end{DoxyCode}


Affichage du code de la fonction permettant le remplissage d\textquotesingle{}une List\+View en fonction d\textquotesingle{}une requête S\+QL \+: 
\begin{DoxyCode}
1 public void listViewStagesFormations(string query, ListView p\_listView)
2         \{
3             // allow reload without superposition
4             p\_listView.Items.Clear();
5             //Open connection
6             if (this.OpenConnection() == true)
7             \{
8                 //Create Command
9                 MySqlCommand cmd = new MySqlCommand(query, connection);
10                 //Create a data reader and Execute the command
11                 MySqlDataReader dataReader = cmd.ExecuteReader();
12 
13                 //Read the data and store them in the list
14                 while (dataReader.Read())
15                 \{
16                     ListViewItem item = new ListViewItem(dataReader["formations\_intitule"].ToString());
17                     item.SubItems.Add(dataReader["associations\_nom"].ToString());
18                     item.SubItems.Add(dataReader["salles\_nom"].ToString());
19                     item.SubItems.Add(dataReader["stages\_formations\_prix"].ToString());
20                     item.SubItems.Add(dataReader["stages\_formations\_placeRestantes"].ToString());
21                     item.SubItems.Add(dataReader["stages\_formations\_date"].ToString());
22 
23                     p\_listView.Items.Add(item);
24                 \}
25 
26                 //close Data Reader
27                 dataReader.Close();
28 
29                 //close Connection
30                 this.CloseConnection();
31 
32             \}
33             else
34             \{
35                 MessageBox.Show("Une erreur est survenue");
36             \}
37         \}
\end{DoxyCode}


Ainsi il est plus facile de gérer l\textquotesingle{}espace de recherche avancée à partir de certains paramètres



Ainsi pour une nouvelle fourchette de date, il suffit à l\textquotesingle{}utilisateur de choisir une date à partir du {\bfseries Date\+Time\+Picker} et de cliquer sur le bouton {\itshape Rechercher par date}.

Cela aura pour effet d\textquotesingle{}executer le code suivant \+:


\begin{DoxyCode}
1 DBConnect dbConnect = new DBConnect();
2 
3 // récupération des dates 
4 string dateDebutUS = dateTimePickerDebut.Value.Date.ToString("yyyy/MM/dd");
5 string dateFinUS = dateTimePickerFin.Value.Date.ToString("yyyy/MM/dd");
6 
7 // écriture de la requête
8 string query = "SELECT * FROM view\_stages\_formations WHERE stages\_formations\_date >= '" + dateDebutUS + "'
       AND stages\_formations\_date <= '" + dateFinUS + "'";
9 
10 // remplissage de la ListView
11 dbConnect.listViewStagesFormations(query, listViewStagesFormations);
\end{DoxyCode}
 